\newcommand{\classpath}{homework}
\documentclass{\classpath}

\begin{document}

\title{An Elegant Proof}
\class{math1023}
\author{John Smith}
\date{July 31, 2020}

\maketitle

\problem{31}{4} Prove the following theorem:
\begin{theorem} 
    \(2^{1/n}\) is irrational for all integers \(n \geq 3\).
\end{theorem}

\textit{Hint:} 
First recall \textbf{Fermat's Last Theorem.} 
Then, set up a proof by contradiction and prove the original theorem.

\alphnum 
    We recall the following theorem.

    \begin{theorem}\label{fermat}
        (Fermat) The equation \(x^n + y^n = z^n\) has no integer solutions for \(x,y,z \neq 0\) and \(n > 2\).
    \end{theorem}
        
    (The proof is trivial and left as an exercise to the reader)

\alphnum
    \begin{lemma}\label{lem}
        Suppose \(2^{1/n}\) is rational for some integer \(n \geq 3\). 
        Then \(\exists a, b \in \mathbb{Z}^{\neq 0}\) for which \(b^n + b^n = a^n\).
    \end{lemma}
    \begin{proof}
        Let \(2^{1/n}\) be rational for some integer \(n \geq 3\).

        Then, there exist nonzero integers \(a\) and \(b\) such that 
        \begin{align*}
            \frac{a}{b} = 2^{1/n}.
        \end{align*}
        And so
        \begin{align*}
            a^n = 2b^n = b^n + b^n.
        \end{align*}
        This completes the proof.
    \end{proof}

\lt{d} % we are skipping (c)
    We will now prove the original statement.

    \begin{proof}
        Let a counterexample to the original theorem exist. Then, by our lemma in (b), 
        there exist nonzero integers \(a\) and \(b\) where \(b^n + b^n = a^n\).
        However, this equation presents a solution to the equation in part (a) which has no solutions.
        Hence, we conclude that no counterexamples to the theorem exist, and therefore, the theorem is true. 
    \end{proof}

% note: there are better ways to prove the theorem

\end{document}
